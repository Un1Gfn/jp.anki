% https://www.overleaf.com/learn/latex/Counters
% https://www.overleaf.com/learn/latex/Lists

【20220405】
\ltjruby{\TeX}{テ\bc{フ}}・\ltjruby{\LaTeX}{ラテック}・
\ltjruby{組版}{くみはん}・\ltjruby{縦組み}{たてぐみ}・\ltjruby{横組}{よこぐ}み・
\ltjruby{前舌母音}{まえじたぼ\bc{いん}}・\ltjruby{前舌母音}{ぜんぜつぼ\bc{いん}}\\
\ltjruby{無声両唇摩擦音}{むせい・りょうしん・まさつ\bcc{おん}}・
\ltjruby{無声軟口蓋摩擦音}{むせい・なんこうがい・まさつ\bcc{おん}}・
\ltjruby{呼気}{こき}でちょっと\ltjruby{曇}{くも}る\\
互いを\ltjruby{糧}{かて}として生きていくんです・
たっているのは\ltjruby{深淵}{しんえん}の\ltjruby{縁}{ふち}・
\ltjruby{福助頭}{ふくすけあたま}が入っていた古ぼけたアパート・
ところどころに\ltjruby{湿気}{しっき}によるしみがうっすらとついている\\

\ltjruby{誕生}{たんじょう}してから40年以上の\ltjruby{歳月}{\bc{さ}いげつ}が\ltjruby{流}{なが}れ、
現在\ltjruby{主流}{しゅりゅう}の姿になって\\
その\ltjruby{歴史}{れきし}の中でさまざまな\ltjruby{変遷}{へんせん}を辿り\santen\ltjruby{培}{つちか}われた常識的な知識が古くなり\\
その傾向は特にこの数年\ltjruby{顕著}{けんちょ}で\santen
{}○○向けに新しい知識が\ltjruby{啓蒙}{けいもう}\\
\ltjruby{本稿}{ほんこう}の主な\ltjruby{想定読者}{ペルソナ}は\santen
\ltjruby{旧来}{きゅうらい}からの知識\santen{}を\ltjruby{対象}{たいしょ\bc{う}}とします\\
日本語で\ltjruby{文書}{ぶんしょ}作成を行う日本語\ltjruby{話者}{わしゃ}\santen{}日本語とは\ltjruby{無縁}{むえん}の海外○○\\
{}この状況への\ltjruby{救世主}{きゅうせいしゅ}として○○というものが○○\\
\ltjruby{非互換}{ひ{\bf{}ご}かん}・自動で\ltjruby{検知}{けんち}・使い方が少し\ltjruby{特殊}{とくしゅ}\\
〜についても明示することを\ltjruby{要求}{よう\bc{きゅう}}もしくは\ltjruby{推奨}{すい\bc{しょう}}しています\\

〜といえば、旧来は\ltjruby{実質的}{じ\bc{っ}しつてき}に○○と呼ばれる\ltjruby{一揃い}{ひとそろい}文書クラス\\
〜が\ltjruby{標準的}{ひょ\bc{う}じゅんてき}な選択肢で\\
〜の満たすべき条件に\ltjruby{合致}{がっち}しておらず\\
〜してはならないというのが○○\ltjruby{界隈}{かいわい}では事実上のコンセンサス\\
その名の\ltjruby{通り}{\bc{と}おり}○○の\ltjruby{策定}{\bc{さ}くてい}する「○の要件」に\ltjruby{準拠}{じゅんきょ}した○○を目指すもの\\

\ltjruby{備}{そ\bc{な}}える・
\ltjruby{供}{そ\bc{な}}える・
\ltjruby{具}{そ\bc{な}}える・
\ltjruby{聳}{そ\bcc{び}}える・
\ltjruby{欧州}{お\bc{う}しゅう}\\
〜の中でも〜ができるように\ltjruby{拡張}{かくちょう}されたものの\ltjruby{種類}{し\bc{ゅ}るい}は限られていました\\
特に日本語組版に詳しい訳ではない人が\ltjruby{一見}{いっけん}したぐらいではフツーの日本語文書に見える程度の\ltjruby{出来栄}{できば}えに仕上がるのですが、一部の禁則処理(\ltjruby{小書}{こが}き仮名の\ltjruby{行頭}{ぎょうとう}禁則)が副作用なしには実現できなかったり\\

\ltjruby{鬼門}{きもん}・
\ltjruby{三途}{さんず}の\ltjruby{川}{かわ}・
\ltjruby{通俗}{つうぞく}・
\ltjruby{冥途}{めい\bc{ど}}・
\ltjruby{賽河原}{さいのかわら}・
\ltjruby{地蔵菩薩信仰}{じぞうぼさつしんこう}・
\ltjruby{\bc{斎}院}{さいいん}・
\ltjruby{供養}{くよう}・
\ltjruby{供物}{くもつ}・
\ltjruby{儀礼}{ぎれい}・
\ltjruby{供犠}{くぎ}・
\ltjruby{簡単}{かん\bc{た}ん}に・
「旧常識」を\ltjruby{覆}{くつがえ}す・
\ltjruby{記述}{\bc{き}じゅつ}・
\ltjruby{技術}{\bcc{ぎ}じゅつ}・
\ltjruby{記載}{\bc{き}さい}・\ltjruby{搭載}{とうさい}・
必死の\ltjruby{追随}{ついずい}対応\\
\ltjruby{一}{ひと}つ・
\ltjruby{二}{ふた}つ・
\ltjruby{三}{みっ}つ・
\ltjruby{四}{よっ}つ・
\ltjruby{五}{いつ}つ・
\ltjruby{六}{むっ}つ・
\ltjruby{七}{なな}つ・
\ltjruby{八}{やっ}つ・
\ltjruby{九}{ここの}つ・
\ltjruby{十}{とお}\\

\ltjruby{著}{いちじる}しく\ltjruby{煩雑}{はんざつ}だったフォント設定・
これ以上\ltjruby{言及}{げんきゅう}しません\\
\ltjruby{殊}{こと}に近年は\santen\ltjruby{本家}{ほん\bc{け}}の更新が\ltjruby{活発}{かっぱつ}に\santen{}\ltjruby{特筆}{とくひ\bc{つ}}するに\ltjruby{値}{あたい}します(\ltjruby{値}{あたい}する)\\
