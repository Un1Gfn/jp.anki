\documentclass[a4paper]{minimal}

% font (A)
% \usepackage{inputenc}
% \usepackage{font}

% font (B)
\usepackage{xltxtra}
\setmainfont{UD Digi Kyokasho N-R}
% \setmainfont{Noto Sans CJK JP}

\usepackage{ruby}

\begin{document}

	\setlength{\parindent}{0em}

	\setlength{\baselineskip}{1.5em}

	すべての人間は、生れながらにして自由であり、かつ、尊厳と権利とについて平等である。 \\
	人間は、理性と良心とを授けられており、互いに同胞の精神をもって行動しなければならない。 \\

	\ruby{吾輩}{わがはい}は猫である。名前はまだ無い。 \\
	どこで生れたかとんと\ruby{見当}{けんとう}がつかぬ。何でも薄暗いじめじめした所でニャーニャー泣いていた事だけは記憶している。吾輩はここで始めて人間というものを見た。 \\

	ああああああああああああああああああああああああああああああああああああああああああああいいいいいいいいいいいいいいいいいいいいいいいいいいいいいいいいいいいいいいいいいいいいいい \\

	いろはにほへと ちりぬるを \\
	わかよたれそ つねならむ \\
	うゐのおくやま けふこえて \\
	あさきゆめみし ゑひもせす \\

	色は匂へど 散りぬるを \\
	我が世誰ぞ 常ならむ \\
	有為の奥山 今日越えて \\
	浅き夢見じ 酔ひもせず \\

	互いを\ruby{糧}{かて}として生きていくんです \\

	たっているのは深淵の\ruby{縁}{ふち} \\


\end{document}
