% https://www.overleaf.com/learn/latex/Counters
% https://www.overleaf.com/learn/latex/Lists

\ltjruby{\TeX}{テ\bc{フ}}・\ltjruby{\LaTeX}{ラテック}\\

\ltjruby{組版}{くみはん}・\ltjruby{縦組み}{たてぐみ}・横組み\\

\ltjruby{前舌母音}{まえじたぼいん}・\ltjruby{前舌母音}{ぜんぜつぼいん}\\
\ltjruby{無声両唇摩擦音}{むせい・りょうしん・まさつおん}\santen%
\ltjruby{無声軟口蓋摩擦音}{むせい・なんこうがい・まさつおん}\santen%
\ltjruby{呼気}{こき}でちょっと\ltjruby{曇}{くも}る\\

互いを\ltjruby{糧}{かて}として生きていくんです\\
たっているのは深淵の\ltjruby{縁}{ふち}\\

\ltjruby{福助頭}{ふくすけあたま}が入っていた古ぼけたアパート\\
ところどころに\ltjruby{湿気}{しっき}によるしみがうっすらとついている\\

\ltjruby{誕生}{たんじょう}してから40年以上の\ltjruby{歳月}{\bc{さ}いげつ}が\ltjruby{流}{なが}れ、現在\ltjruby{主流}{しゅりゅう}の姿になって\\
その\ltjruby{歴史}{れきし}の中でさまざまな\ltjruby{変遷}{へんせん}を辿り\santen\ltjruby{培}{つちか}われた常識的な知識が古くなり\\
その傾向は特にこの数年\ltjruby{顕著}{けんちょ}で\santen{}○○向けに新しい知識が\ltjruby{啓蒙}{けいもう}\\
\ltjruby{本稿}{ほんこう}の主な\ltjruby{想定読者}{ペルソナ}は\santen{}\ltjruby{旧来}{きゅうらい}からの知識\santen{}を\ltjruby{対象}{たいしょ\bc{う}}とします\\
日本語で\ltjruby{文書}{ぶんしょ}作成を行う日本語\ltjruby{話者}{わしゃ}\santen{}日本語とは\ltjruby{無縁}{むえん}の海外○○\\
{}この状況への\ltjruby{救世主}{きゅうせいしゅ}として○○というものが○○\\
\ltjruby{非互換}{ひ{\bf{}ご}かん}・自動で\ltjruby{検知}{けんち}・使い方が少し\ltjruby{特殊}{とくしゅ}・\ltjruby{推奨}{すいしょ\bc{う}}され\\
使用するエンジンについても明示することを\ltjruby{要求}{よう\bc{きゅう}}もしくは\ltjruby{推奨}{すいしょ\bc{う}}しています\\

日本語文書用の文書クラスといえば、旧来は\ltjruby{実質的}{じ\bc{っ}しつてき}にjsclassesと呼ばれる\ltjruby{一揃い}{ひとそろい}文書クラス\santen{}が\ltjruby{標準的}{ひょ\bc{う}じゅんてき}な選択肢で\\

\ltjruby{標準的}{ひょ\bc{う}じゅんてき}な日本語組版の満たすべき条件に\ltjruby{合致}{がっち}しておらず\\
○○してはならないというのがTeX\ltjruby{界隈}{かいわい}では事実上のコンセンサス

その名の\ltjruby{通り}{\bc{と}おり}W3Cの\ltjruby{策定}{\bc{さ}くてい}する「日本語組版の要件 (JLREQ)」に\ltjruby{準拠}{じゅんきょ}した文書作成を目指すもので\\

\ltjruby{備}{そ\bc{な}}える・%
\ltjruby{供}{そ\bc{な}}える・%
\ltjruby{具}{そ\bc{な}}える・%
\ltjruby{聳}{そ\bcc{び}}える\\

\ltjruby{欧州}{お\bc{う}しゅう}\santen%
LaTeXエンジンと呼ばれるものの中でも日本語ができるように 拡張.かくちょう されたものの 種類.しゅるい は限られていました
